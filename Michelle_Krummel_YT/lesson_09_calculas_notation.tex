\documentclass[11pt]{article}

\usepackage[margin=0.75in, paperwidth=8.5in, paperheight=11in]{geometry}

\begin{document}


The function $f(x)=(x-3)^2+\frac{1}{2}$ has domain $\mathrm{D}_f:(-\infty, \infty)$ and range $\mathrm{R}_f:\left[\frac{1}{2}, \infty\right) $
\\
here above, "infty" is used for infinity, "mathrm" is used so that "D" is not displayed as italics under math mode.\\



$\lim \limits_{x \to a}$\\
if we do not use "limits" then the subscript will appear on the right side, not under "lim"\\

$\lim \limits_{x \to a^-}f(x)$\\
if we want, as limit approaches a from the right hand side\\
then we will put "caret plus"\\
if we want as, limit approches a from the left hand side\\
then we will put "caret minus"\\

$\lim \limits_{x \to a} \frac{f(x)-f(a)}{x-a}=f'(a)$\\
in the above equation, the fraction takes up same space as the rest of the line\\
so we will use "displaystyle" to make the fraction look bigger\\

$\displaystyle{\lim \limits_{x \to a} \frac{f(x)-f(a)}{x-a}=f'(a)}$\\


INTEGRAL\\

$\int \sin x \,dx=-cos x + C$\\
backslash int is for integral\\

$\displaystyle{\int \sin x \,dx=-cos x + C}$\\
if we use "displaystyle", then the integral symbol appears bigger\\


DEFINITE INTEGRAL\\

$\int_a^b$\\
here the limits a and b appear to the right of the integral symbol\\

$\int \limits_a^b$\\
here the limits a and b appear to the bottom and top of the integral symbol\\

if you want the integral symbol to be elongated then use "displaystyle"\\
$\displaystyle{\int_a^b}$\\
$\displaystyle{\int \limits_a^b}$\\

if our limit is more than 2 caharacters, then we cannot just use a caret\\
it will not be displayed properly like below\\
$\displaystyle{\int \limits_2a^b}$\\

so to fix this we should put the upper and lower limits inside curly brackets\\
$\displaystyle{\int \limits_{2a}^{b}}$\\

lets take integral of something\\
$\displaystyle{\int \limits_{a}^{b}x^2 \,dx=[\frac{x^3}{3}]_{a}^{b}}$\\
here the brackets dont look good\\
the brackets have to be expanded\\

$\displaystyle{\int \limits_{a}^{b}x^2 \,dx=\left[\frac{x^3}{3}\right]_{a}^{b}}$\\
that looks much better\\
we added backslash left bracket symbol\\
and backslash right bracket symbol\\

lets finish evaluating our equation\\
$\displaystyle{\int \limits_{a}^{b}x^2 \,dx=\left[\frac{x^3}{3}\right]_{a}^{b}=\frac{b^3}{3}-\frac{a^3}{3}}$\\


SUMMATION NOTATION:\\

$\sum$\\
this looks smaller\\

$\displaystyle{\sum}$\\
this looks much better, larger\\


$\displaystyle{\sum \limits_{n=1}^{\infty}}$\\

$\displaystyle{     \sum \limits_{n=1}^{\infty}  ar^n=a+ar+ar^2+...+ar^n   }$\\
the above puts dots not aligned properly\\
so we will use "cdot" for single dot, or "cdots" for three dots, which will be aligned properly\\

$\displaystyle{     \sum \limits_{n=1}^{\infty}  ar^n=a+ar+ar^2+  \cdots   +ar^n   }$\\



$\displaystyle{   \int_a^b f(x) \,dx=\lim \limits_{x \to \infty} \sum \limits_{k=1}^{n} f(x_k) \cdot \Delta x  }$\\
"backslash delta" is for small delta greek symbol\\
"backslash Delta" is for big delta greek symbol\\


VECTORS:\\


$\vec{v}=v_1 \vec{i}+v_2 \vec{j}=\langle v_1,v_2 \rangle$\\
"backslash vec" for vector\\
"backslash langle" for left angular bracket\\
"backslash rangle" for right angular bracket\\




\end{document}

\documentclass[11pt]{article}
\usepackage{hyperref}

\title{My \LaTeX\ document}
\author{Kamaljeet Singh}
\date{July 26, 2020}
\date{\today}


\begin{document}

\tableofcontents
\maketitle


The stuff after the documentclass and before begin document is called as the preamble.\\

backslash maketitle: will print the title, author, date that is mentioned in the preamble.\\

for date, we can manually put any text\\
or we can add backslash today, so when we compile the document it will check and add that date\\

the backslash tableofcontents will create the table of contents by referring to the sections and subsections in the document\\
when we compile the document then it will show the heading "contents" and no actual contents under it\\
we have to compile the document two times for the table of contents to actually show up\\

TOPIC: text and document formatting\\

This will produce \textit{italicized} text.

This will produce \textbf{bold faced} text.

This will produce \textsc{small caps} text.

This will produce \texttt{typewriter text font} text.\\
typewriter font is monospaced.

use package hyperref to add url reference\\
it will automatically add a clickable url with typewriter font\\
Please visit abc's website at \url{https://abc.com}

\vspace{1cm}

using href we can have some clickable text to refer to some website url\\
Please visit abc's website at \href{https://abc.com}{My Website}

\vspace{1cm}

Please excuse my dear aunt Sally.\\

make --dear aunt Sally-- really large\\

Please excuse my \begin{large}dear aunt Sally\end{large}.

Please excuse my \begin{Large}dear aunt Sally\end{Large}.

Please excuse my \begin{huge}dear aunt Sally\end{huge}.

Please excuse my \begin{Huge}dear aunt Sally\end{Huge}.

Please excuse my \begin{normalsize}dear aunt Sally\end{normalsize}.\\
this is the same as the first line without any size specified.\\

\vspace{1cm}

now lets do smaller than the normal size

Please excuse my \begin{small}dear aunt Sally\end{small}.

Please excuse my \begin{scriptsize}dear aunt Sally\end{scriptsize}.

Please excuse my \begin{tiny}dear aunt Sally\end{tiny}.

\vspace{2cm}

how to justify text\\

\begin{center}This line is centered.\end{center}

\begin{flushleft}This line is left-justified.\end{flushleft}

\begin{flushright}This line is right-justified.\end{flushright}

\vspace{1cm}

different way to justify text\\
backslash centering: will make all the text centered that comes after it\\

\centering
This line is centered.\\
This line is centered.\\
This line is centered.\\

\vspace{1cm}

backslash Large: will make all the text Large that comes after it\\

\Large
This line is large.\\
This line is large.\\
This line is large.\\

\vspace{1cm}

backslash tiny: will make all the text tiny that comes after it\\

\tiny
This line is tiny.\\
This line is tiny.\\
This line is tiny.\\

\vspace{1cm}

\normalsize
\raggedright

TOPIC: sections and subsections\\
below will show sections and subsections with numbering\\
if we want to hide the numbering then we can put asterisk (*) just after the words section and subsection\\

\section{Linear Functions}
  \subsection{Slope-Intercept Form}
    \subsubsection{Example 1:}
    \subsubsection{Example 2:}
  \subsection{Standard Form}
  \subsection{Point-Slope Form}
\section{Quadratic Functions}
  \subsection{Vertex Form}
  \subsection{Standard Form}
  \subsection{Factored Form}



\end{document}

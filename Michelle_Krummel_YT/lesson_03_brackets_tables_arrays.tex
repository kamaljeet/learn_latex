\documentclass[24pt]{article}
\pagestyle{empty}
\usepackage{amsmath, amssymb, amsfonts}
\usepackage{float}
\parindent 0px

\begin{document}

TOPIC: brackets, tables, arrays

parindent above is to indent paragraphs by some pixels

paranthesis and square brackets are easy, just type them as it is\\

The distributive property states that $a(b+c)=ab+ac$, for all $a, b, c \in \mathbb{R}$.\\[6pt]
The equivalence class of $a$ is $[a]$.\\

curly brackets are not so easy\\

this below will not show curly brackets\\
curly brackets are special symbols which are part of the syntax\\
The set $A$ is defined to be ${1, 2, 3}$.\\[6pt]

backslash has to be added to display curly brackets in the output\\
The set $A$ is defined to be $\{1, 2, 3\}$.\\[6pt]

another symbol that is reserved and is part of the syntax is the dollar symbol\\
so a backslash has to be added\\
The movie ticket costs $\$11.50$\\[6pt]

This equation below bad, the paranthesis is not tall enough to cover whats inside them\\
$$2(\frac{1}{x^2-1})$$

so it has to be written like below, it looks much better\\
the paranthesis get expanded to appropriate size\\
$$2\left(\frac{1}{x^2-1}\right)$$
$$2\left[\frac{1}{x^2-1}\right]$$
$$2\left\{\frac{1}{x^2-1}\right\}$$

for angular brackets below
$$2\left \langle   \frac{1}{x^2-1}\right  \rangle  $$

for absolute value expression
$$2\left  |   \frac{1}{x^2-1}\right  |  $$

if we need bracket only in one side, we cannot use only backslash left or right\\
we have to mention both, there should be a matching one for the other\\
we want to display a single bracket in the right\\
so mention a backslash left with a period\\
then it will not display that\\
$$\left.  \frac{dy}{dx}\right | _{x=1}$$

paranthesis will automatically adjust according to the content inside\\
$$\left(  \frac{1}{1+ \left(  \frac{1}{1+x}  \right)}  \right)$$\\[20pt]




Tables:\\

below 6 c's means 6 columns which are centered aligned\\
6 l's mean 6 columns that are left aligned\\
6 r's mean 6 columns that are right aligned\\
we can also do combination like    ccllrr   for alignment of each column\\
we need ampersand symbol to separate data into each column\\[6pt]
\begin{tabular}{cccccc}
x & 1 & 2 & 3 & 4 & 5
\end{tabular}

\vspace{1cm}

if we want vertical lines in between columns then put a pipe symbol\\
two backslashes at the of the row indicates that this is end of row\\
backslash hline give horizontal line for the table\\[6pt]
\begin{tabular}{|c|c|c|c|c|c|}
\hline
$x$ & 1 & 2 & 3 & 4 & 5 \\ \hline
$f(x)$ & 10 & 11 & 12 & 13 & 14 \\ \hline
\end{tabular}

\vspace{1cm}
vspace to add some vertical space
\vspace{1cm}

we can use double slash to separate the label from the values in the table\\
\begin{tabular}{|c||c|c|c|c|c|}
\hline
$x$ & 1 & 2 & 3 & 4 & 5 \\ \hline
$f(x)$ & 10 & 11 & 12 & 13 & 14 \\ \hline
\end{tabular}

\vspace{1cm}

here we use begin table, begin tabular\\
and also end them\\
after begin table, we use capital H inside brackets to display the table right here\\
otherwise the compiler will decide which is the best place to display the table\\
to use this feature, we have used the package float in the beginning of this code\\
however this does not display the fraction correctly\\
\begin{table}[H]
\begin{tabular}{|c||c|c|c|c|c|}
\hline
$x$ & 1 & 2 & 3 & 4 & 5 \\ \hline
$f(x)$ & $\frac{1}{2}$ & 11 & 12 & 13 & 14 \\ \hline
\end{tabular}
\end{table}


\vspace{1cm}

to display the fraction correctly we have to add some extra space by using\\
def arraystretch and some number for the extra space\\
\begin{table}[H]
\def\arraystretch{1.8}
\begin{tabular}{|c||c|c|c|c|c|}
\hline
$x$ & 1 & 2 & 3 & 4 & 5 \\ \hline
$f(x)$ & $\frac{1}{2}$ & 11 & 12 & 13 & 14 \\ \hline
\end{tabular}
\end{table}


\vspace{1cm}

we can enter some caption to our table to the beginning or end\\
also we can center our table\\
\begin{table}[H]
\centering
\def\arraystretch{1.8}
\begin{tabular}{|c||c|c|c|c|c|}
\hline
$x$ & 1 & 2 & 3 & 4 & 5 \\ \hline
$f(x)$ & $\frac{1}{2}$ & 11 & 12 & 13 & 14 \\ \hline
\end{tabular}
\caption{These values represent the function $f(x)$.}
\end{table}


\vspace{1cm}

we can put a sentance also inside a table\\
\begin{table}[H]
\centering
\caption{The relationship between $f$ and $f'$}
\def\arraystretch{1.8}
\begin{tabular}{|c|c|}
\hline
$f(x)$ & $f'(x)$ \\ \hline
$x>0$ & The function $f(x)$ is increasing. \\ \hline
\end{tabular}
\end{table}


\vspace{1cm}

how to put long sentences\\
this below sentance gets cut off\\
\begin{table}[H]
\centering
\caption{The relationship between $f$ and $f'$}
\def\arraystretch{1.8}
\begin{tabular}{|l|l|}
\hline
$f(x)$ & $f'(x)$ \\ \hline
$x>0$ & The function $f(x)$ is increasing. The function $f(x)$ is increasing. The function $f(x)$ is increasing. The function $f(x)$ is increasing. The function $f(x)$ is increasing. The function $f(x)$ is increasing. The function $f(x)$ is increasing. The function $f(x)$ is increasing. \\ \hline
\end{tabular}
\end{table}


\vspace{1cm}

how to put long sentences\\
we will mention a p to indicate a paragraph and then some size like\\
2in for inches or 2cm for centimeters\\
\begin{table}[H]
\centering
\caption{The relationship between $f$ and $f'$}
\def\arraystretch{1.8}
\begin{tabular}{|l|p{2in}|}
\hline
$f(x)$ & $f'(x)$ \\ \hline
$x>0$ & The function $f(x)$ is increasing. The function $f(x)$ is increasing. The function $f(x)$ is increasing. The function $f(x)$ is increasing. The function $f(x)$ is increasing. The function $f(x)$ is increasing. The function $f(x)$ is increasing. The function $f(x)$ is increasing. \\ \hline
\end{tabular}
\end{table}


\vspace{2cm}

Equation Arrays:\\

\vspace{1cm}

we can use \\
backslash begin and inside curly braces, eqnarray\\
then backslash end and inside curly braces, eqnarray\\

but there is a different way to do it\\
inside align we are automatically in math mode, so we dont need dollar\\
here if we need to display text then we have to use backslash text\\
if we need to display space then use, backslash comma\\
\begin{align}
5x^2\, \text{ place your words here}
\end{align}

\vspace{1cm}

equations are being numbered\\
but equations are not aligned at the equal sign\\
\begin{align}
5x^2-9=x+3\\
5x^2-x-12=0
\end{align}

\vspace{1cm}

add ampersand in front of equal sign to align the equations according to the equal sign\\
\begin{align}
5x^2-9&=x+3\\
5x^2-x-12&=0\\
&=12+x-5x^2
\end{align}


\vspace{1cm}

use begin align asterix to turn off the equation numbering\\
it has to be added to both begin and end\\
\begin{align*}
5x^2-9&=x+3\\
5x^2-x-12&=0\\
&=12+x-5x^2
\end{align*}

\vspace{1cm}

if we again use begin align, the numbering continues where it left off\\
it is not hiding the numbering with asterix, it just turns off\\
\begin{align}
5x^2-9&=x+3\\
5x^2-x-12&=0\\
&=12+x-5x^2
\end{align}






\end{document}
